\documentclass[a4paper]{article}
\usepackage[T1]{polski}
\usepackage[utf8]{inputenc}
\usepackage{inputenc}
\usepackage{listings}
\usepackage{color}

\definecolor{dkgreen}{rgb}{0,0.6,0}
\definecolor{gray}{rgb}{0.5,0.5,0.5}
\definecolor{mauve}{rgb}{0.58,0,0.82}

\lstset{frame=tb,
  language=Lisp,
  aboveskip=3mm,
  belowskip=3mm,
  showstringspaces=false,
  columns=flexible,
  basicstyle={\small\ttfamily},
  numbers=none,
  numberstyle=\tiny\color{gray},
  keywordstyle=\color{blue},
  commentstyle=\color{dkgreen},
  stringstyle=\color{mauve},
  breaklines=true,
  breakatwhitespace=true,
  tabsize=3
}


\title{Metody Programowania lista 1}
\author{Szymon Kopa}
\date{20 lutego 2018}

\begin{document}
\maketitle
\clearpage

\section{Ćwiczenie 1}
\subsection{}
\begin{lstlisting}
>10
10
\end{lstlisting}
\subsection{}
\begin{lstlisting}
>(+ 5 3 4)
12
\end{lstlisting}
\subsection{}
\begin{lstlisting}
>(- 9 1)
8
\end{lstlisting}
\subsection{}
\begin{lstlisting}
>(/ 6 2)
3
\end{lstlisting}
\subsection{}
\begin{lstlisting}
>(+ (* 2 4) (- 4 6) )

6
\end{lstlisting}
\subsection{}
\begin{lstlisting}
>(define a 3)
>( define b (+ a 1) )
>(+ a b (* a b ) )

19
\end{lstlisting}
\subsection{}
\begin{lstlisting}
>(= a b )

#f
\end{lstlisting}
\subsection{}
\begin{lstlisting}
>( if ( and (> b a ) (< b (* a b ) ) ) b a)

4
\end{lstlisting}
\subsection{}
\begin{lstlisting}
>( cond [(= a 4) 6]
[(= b 4) (+ 6 7 a ) ]
[ else 25])

16
\end{lstlisting}
\subsection{}
\begin{lstlisting}
>(+ 2 ( if (> b a ) b a ) )

6
\end{lstlisting}
\subsection{}
\begin{lstlisting}
>(* ( cond [( > a b ) a ]
[(< a b ) b ]
[ else
-1])
(+ a 1) )

16
\end{lstlisting}

\section{Ćwiczenie 2}
\begin{center}
    $$ \frac{5+4+(2-(3-(6+\frac{4}{5})))}{3(6-2)(2-7)} $$
\begin{lstlisting}
    (/ ({+ 5 4 (- 2 {- 3 (+ 6 {/ 4 5})})}) {* 3 (- 6 2 ) (- 2 7 )})
\end{lstlisting}
\end{center}
\section{Ćwiczenie 3}
\clearpage
\section{Ćwiczenie 4}
\begin{lstlisting}
    #lang racket
    (define (sum-of-squares a b)
      (+ (* a a) (* b b)))
    (define (max a b)
      (cond [(> a b) a]
            [else b]))
    (define (sum-of-squares-of-2-largest a b c)
      (cond [(> a b) (sum-of-squares a (max b c))]
            [else (sum-of-squares b (max a c))]))
\end{lstlisting}
\section{Ćwiczenie 5}

\end{document}
